\chapter*{Abstract}
%\addcontentsline{toc}{chapter}{Abstract}
\vspace{-1em}

Buildings are the largest consumers of energy worldwide. Within a building, heating, ventilation and air-conditioning (HVAC) systems consume the most energy, leading to trillion dollars of electrical expenditure worldwide each year. With rising energy costs and increasingly stringent regulatory environments, improving the energy efficiency of HVAC operations in buildings has become a global concern. From a short-term economic point-of-view, with over 100 billion dollars in annual electricity expenditures, even a small percentage improvement in the operation of HVAC systems can lead to significant savings. From a long-term point-of-view, the need of fostering a smart and sustainable built environment calls for the development of innovative HVAC control strategies in buildings.

In this thesis, we look at the potential for integrating building operations with room booking and occupancy scheduling. More specifically, we explore novel approaches to reduce HVAC consumption in commercial buildings, by jointly optimising the occupancy scheduling decisions (e.g. the scheduling of meetings, lectures, exams) and the building's occupancy-based HVAC control. Our vision is to integrate occupancy scheduling with HVAC control, in such a way that the energy consumption is reduced, while the occupancy thermal comfort and scheduling requirements are addressed. We identify four unique research challenges which we simultaneously tackle in order to achieve this vision, and which form the major contributions of this thesis.

% We address this question in four parts that come together to form a complete solution:
%   the formulation of a joint hvac control and occupancy scheduling model, 
%   the development of a scalable joint model to solve large problems, 
%   the introduction of an online joint model that is capable of coping with impromptu meeting requests, and
%   the integration of adaptive temperature control that further improves on the energy savings.

Our first contribution is an integrated model that achieves \textsl{high efficiency in energy reduction} by fully exploiting the capability to coordinate HVAC control and occupancy scheduling. The core component of our approach is a mixed-integer linear programming (MILP) model which optimally solves the joint occupancy scheduling and occupancy-based HVAC control problem. Existing approaches typically solve these subproblems in isolation: either scheduling occupancy given conventional control policies, or optimising HVAC control using a given occupancy schedule. From a computation standpoint, our joint problem is much more challenging than either, as HVAC models are traditionally non-linear and non-convex, and scheduling models additionally introduce discrete variables capturing the time slot and location at which each activity is scheduled. We find that substantial reduction in energy consumption can be achieved by solving the joint problem, compared to the state of the art approaches using heuristic scheduling solutions and to more na\"{\i}ve integrations of occupancy scheduling and occupancy-based HVAC control.

Our second contribution is an approach that scales to large occupancy scheduling and HVAC control problems, featuring hundreds of activity requests across a large number of offices and rooms. This approach embeds the integrated MILP model into Large Neighbourhood Search (LNS). 
%a Large Neighbourhood Search (LNS) approach that is capable of handling hundreds of activity requests across a large number of offices and rooms. 
%We embed the integrated MILP model into LNS. 
LNS is used to destroy part of the schedule and MILP is used to repair the schedule so as to minimise energy consumption. Given sets of occupancy schedules with different constrainedness and sets of buildings with varying thermal response, our model is sufficiently \textsl{scalable} to provide instantaneous and near-optimal solutions to problems of realistic size, such as those found in university timetabling.

The third contribution is an \textsl{online} optimisation approach that models and solves the online joint HVAC control and occupancy scheduling problem, in which activity requests arrive dynamically. This online algorithm greedily commits to the best schedule for the latest activity requests, but revises the entire future HVAC control strategy each time it considers new requests and weather updates. We ensure that whilst occupants are instantly notified of the scheduled time and location for their requested activity, the HVAC control is constantly re-optimised and adjusted to the full schedule and weather updates. We demonstrate that, even without prior knowledge of future requests, our model is able to produce energy-efficient schedules which are close to the clairvoyant solution. 

Our final contribution is a \textsl{robust} optimisation approach that incorporates adaptive comfort temperature control into our integrated model. We devise a robust model that enables flexible comfort setpoints, encouraging energy saving behaviors by allowing the occupants to indicate their thermal comfort flexibility, and providing a probabilistic guarantee for the level of comfort tolerance indicated by the occupants. We find that dynamically adjusting temperature setpoints based on occupants' thermal acceptance level can lead to significant energy reduction over the conventional fixed temperature setpoints approach.
%produce substantially more feasible solutions than the conventional fixed temperature setpoints approach in an online manner. 

Together, these components deliver a complete optimisation solution that is efficient, scalable, responsive and robust for online HVAC-aware occupancy scheduling in commercial buildings. 



% AAAI-15
%Energy consumption in commercial and educational buildings is impacted by group activities such as meetings, workshops, classes and exams, and can be reduced by scheduling these activities to take place at times and locations that are favorable from an energy standpoint. This paper improves on the effectiveness of energy-aware room-booking and occupancy scheduling approaches, by allowing the scheduling decisions to rely on an explicit model of the building's occupancy-based HVAC control.  The core component of our approach is a mixed-integer linear programming (MILP) model which optimally solves the joint occupancy scheduling and occupancy-based HVAC control problem.  To scale up to realistic problem sizes, we embed this MILP model into a large neighbourhood search (LNS).  We obtain substantial energy reduction in comparison with occupancy-based HVAC control using arbitrary schedules or using schedules obtained by existing heuristic energy-aware scheduling approaches.

% CPAIOR-15
%One of the main inefficiencies in building management systems is the widespread use of schedule-based control when operating heating, ventilation and air conditioning (HVAC) systems. HVAC systems typically operate on a pre-designed schedule that heats or cools rooms in the building to a set temperature even when rooms are not being used. Occupants, however, influence the thermal behavior of buildings. As a result, using occupancy information for scheduling meetings to occur at specific times and in specific rooms has significant energy savings potential. As shown in Lim et al. \cite{lim15hvac}, combining HVAC control with meeting scheduling can lead to substantial improvements in energy efficiency. We extend this work and develop an approach that scales to larger problems by combining mixed integer programming (MIP) with large neighbourhood search (LNS). LNS is used to destroy part of the schedule and MIP is used to repair the schedule so as to minimise energy consumption. This approach is far more effective than solving the complete problem as a MIP problem. Our results show that solutions from the LNS-based approach are up to 36\% better than the MIP-based approach when both given 15 minutes.

% CP-16
%Heating, ventilation and air-conditioning (HVAC) is the lar\-gest consumer of electricity in commercial buildings. Consumption is impacted by group activities (e.g. meetings, lectures) and can be reduced by scheduling these activities at times and locations that minimise HVAC utilisation. However, this needs to preserve occupants' thermal comfort and be responsive to dynamic information such as new activity requests and weather updates. This paper presents an \textsl{online} HVAC-aware occupancy scheduling approach which models and solves a joint HVAC control and occupancy scheduling problem. Our online algorithm greedily commits to the best schedule for the latest activity requests and notifies the occupants immediately, but revises the entire future HVAC control strategy each time it considers new requests and weather updates. In our experiments, the quality of the solution obtained by this approach is within 1\% of that of the clairvoyant solution.  We incorporate \textsl{adaptive comfort temperature control} into our model, encouraging energy saving behaviors by allowing the occupants to indicate their thermal comfort flexibility. In our experiments, the integration of adaptive temperature control further generates up to 12\% of energy savings when a reasonable thermal comfort flexibility is provided

%%% Local Variables: 
%%% mode: latex
%%% TeX-master: "paper"
%%% End: 